%%%%%%%%%%%%%%%%%%%%%%%%%%%%%%%%%%%%%%%%%%%%%%%%
\section{Conclusiones} % NOMBRE DE LA SECCIÓN
\label{sec:conclusiones} % ETIQUETA
%%%%%%%%%%%%%%%%%%%%%%%%%%%%%%%%%%%%%%%%%%%%%%%%
\thispagestyle{plain}

Se han conseguido avances importantes en cada uno de los grupos de trabajo.
Tenemos una idea clara de los sensores y componentes requeridos y como
implementarlos con Arduino, incluyendo el módulo Esp32, que se va a usar para la
conexión con la página web. La página web ya cuenta con un diseño bastante
minimalista, con la intención de hacer una experiencia de usuario bastante
agradable y sencilla. Para la creación de esta se tiene en mente la posibilidad
de usar Amazon Web Service (AWS). A demás un modelo 3D bastante logrado, el
cual se planea traer a la realidad con el uso de impresoras 3D y hace una muy
buena representación de como esperamos que se vea el producto final.

En consecuencia, es notable como en este primer periodo el proyecto ha tomado
forma, hemos conseguido bastantes recursos informativos en los que podemos
apoyarnos para dar paso a una siguiente fase más práctica, pues también se ha
mostrado una buena participación y actitud de parte del grupo.
