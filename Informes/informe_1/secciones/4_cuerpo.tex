%%%%%%%%%%%%%%%%%%%%%%%%%%%%%%%%%%%%%%%%%%%%%%%%%
\section{Cuerpo} % NOMBRE DE LA SECCIÓN
\label{sec:cuerpo} % ETIQUETA
%%%%%%%%%%%%%%%%%%%%%%%%%%%%%%%%%%%%%%%%%%%%%%%%%
\thispagestyle{plain}

\subsection{Metodología}
\label{ssec:metodologia}

\subsubsection{Materiales}

\noindent La lista de materiales para el desarrollo del prototipo se encuentra
en del anexo \ref{a:materiales1} al anexo \ref{a:materiales2}. Principalmentelos deben llegar a la siguiente persona:

\begin{table}[htpb]
    \label{tab:label}
        \begin{tabular}{r|l}
            Contacto:& Luisa Fernanda Salcedo Cortes\\
            Correo:& luisa.salcedo@urosario.edu.co\\
            Celular:& 3014887712\\
            Dirección:& Carrera 67 #106-60\\
                      & Torre 4 apartamento 603\\
                      & Conjunto Quintanilla de la flora\\
        \end{tabular}
\end{table}

Pero en el caso de que haya disponibilidad para duplicar la lista (por lo menos la del anexo \ref{a:materiales2}) le deben llegar a:

\begin{table}[htpb]
    \label{tab:label}
        \begin{tabular}{r|l}
            Contacto:& Nicolas Otero\\
            Correo:& nicolas.otero@urosario.edu.co\\
            Celular:& 3229083234\\
            Dirección:& Cra7b\#138-79\\
                      &Portobelo 502\\
        \end{tabular}
\end{table}


\subsubsection{Metodología de Trabajo}

\noindent El grupo de trabajo del proyecto Agrotech determino su metodología en
las siguientes áreas de trabajo hardware, software, reportes e impresión 3D. El
área de hardware se encarga de la estructuración y modelación del proyecto en
escala e implementación de circuito electrónico, el área de software se encarga
de la estructuración y desarrollo del aplicativo web para su respectivo enlace
con la estructura a realizar, en cuanto al área de redacción se encarga de la
parte investigativa, consultas y desarrollo del trabajo a presentar y en el
área de impresión 3D se encarga de la modelación acompañado con el área de
hardware para detallar medidas del prototipo y realizar los modelos necesarios
para imprimir piezas que el proyecto requiere, adicionalmente el proyecto
cuenta con una líder la cual representa al grupo en reuniones de avances, envía
la información correspondiente y se encarga de administrar que todos los
integrantes del proyecto tengan una actividad asignada. Cada grupo se determinó
mediante las habilidades de cada persona y todo el proyecto se ha desarrollado
bajo ideas conjuntas e involucra a todos sus integrantes en la toma de
decisiones y así mismo para la solución de problemas que se pueden presentar en
el proyecto.

\subsubsection{Datos de Investigación}

\noindent Al momento de decidir la estructura del sistema de cultivo se planteó
desde dos conceptos fundamentalmente distintos. El primero de estos era uno
bastante tradicional, donde las semillas se plantan en tierra y con ayuda de
censores se maneja la distribución de agua, el control de luz artificial y el
nivel de ph en la tierra. La segunda fue una opción menos convencional, se
trata de cultivos hidropónicos donde la cantidad de terreno orgánico muy
reducido con respecto de la primera opción, ya que las semillas solo tienen
contacto con una pequeña cantidad de tierra, permitiendo una mejor oxigenación
(\cite{usatoday2019}); con esté sistema se considera pertinente monitorear no
solamente la luz y el ph, sino que también la temperatura ambiente ya que en
los sistemas hidropónicos es fundamental manejar el riego partiendo de las
condiciones ambientales(\cite{cabazas2020}).

\noindent Por lo innovador del sistema y los beneficios que brinda el no
utilizar tierra, se optó por la segunda opción. Ya en el proceso de
investigación sobre la realización del cultivo hidropónico se encontró la
opción de hacer el modelo no basándonos en el cultivo hidropónico horizontal
que es un poco más convencional sino que se construyera de forma vertical. Se
decidió optar por el diseño vertical, ya que no solo es más adecuado para
espacios reducidos sino que además es más eficiente en cuestión de producción
de alimentos (\cite{doi:10.1002/fes3.83}).

\subsubsection{Herramientas de Diseño}

\begin{itemize}
    \item \textbf{Figma}: \\
        Para el diseño del \textit{wireframe} de la página web se utilizó la
        herramienta gratuita \textbf{Figma}. A la cual se puede acceder con la
        dirección url: \textit{\href{https://www.figma.com}{https://www.figma.com}}. Es bastante intuitiva,
        fácil de utilizar y permite simular el \textit{flow} de la página web
        con gran agilidad. La herramienta se utilizó desde \textit{firefox
        80.0.1}  y no se necesito instalar nada adicional.
    \item \textbf{Modo Creative 3D Modeling Software}:\\
        El software que se implementó para el diseño 3D del sketch  fue Modo
        Creative 3D Modeling Software. Este programa pagado es compatible con
        las marcas Windows, Mac OS y Linux, para el uso de la modelación 3D.
        Este programa no cuenta con formato web, lo que implica instalarlo con
        los siguientes requisitos de instalación(recomendados):
        \begin{description}
            \item[Procesador] Intel, Core i3 en adelante.
            \item[Almacenamiento] Mínimo 10GB de espacio en el disco duro para
                contenido total.
            \item[RAM] Mínimo 4GB.
            \item[Resolución] 1280 x 800 pixeles.
            \item[Tarjeta Grafica] Mínimo 1GB.
        \end{description}
         Se optó por utilizar este programa ya que se basa en modelación
         procedimental no destructiva, es decir, permite modificar, editar y
         hasta revertir lo que se conoce como <<meshes>>  de forma intuitiva.
         Adicionalmente, los objetos modelados se basan en polígonos simples
         que se pueden alterar para crear ciertas figuras, siempre en base a
         las dimensiones propuestas por el usuario, lo que implica un prototipo
         preciso y a escala.

         Otro aspecto fundamental por el cual se usó este programa fue su
         integración con la impresión 3D. A la hora de imprimir, es de suma
         importancia asegurarse de que el modelo sea estanco para así evitar
         problemas estructurales con el producto final. En Modo 3D, al modelar,
         el programa automáticamente avisa si se está cumpliendo este factor, y
         de esa forma, ayudar a detectar errores y solucionarlos rápidamente.

         En nuestro caso, los elementos expuestos anteriormente contribuyeron
         mucho a la creación de nuestro modelo ya que los componentes están
         hechos a base de dimensiones a escala y su integración se hace
         fácilmente. Además, Modo 3D genera imágenes del modelo interactivas,
         ayudando a la visualización de ella para tener un concepto claro del
         prototipo.
\end{itemize}

\thispagestyle{plain}
\subsection{Diseño}
\label{ssec:diseno}

\subsubsection{Estructura física}

Para el prototipo usaremos yee's de PVC con dimensiones de 2 pulgadas y uniones
de 2 y 1.5 pulgadas respectivamente para debido a que están hechas para un
contacto constante con el agua, algo que será fundamental dentro de la torre
hidropónica y ayudará a evitar accidentes hídricos.

Como base usamos un balde de pintura que posee las propiedades de un plástico
grueso que se incorpora con los demás componentes mediante el proceso de
soldadura del PVC. Debido a que la implementación del sistema hídrico será
cerrada, la tapa superior del módulo estará impresa en 3D para una distribución
del agua de manera uniforme y control térmico de las plantas. Luego, en la
parte técnica usaremos una bomba de agua sumergible por facilidades del diseño,
que estará conectada a un replay de 5V que a su vez ira con el Arduino que será
el que posee conexiones con sensores térmicos, de humedad y de
fotosensibilidad.
Para la conexión con la página  web, se usará un módulo sp32 de wifi y bluetooth.

Concorde a los diferentes ángulos del sketch, los elementos son los siguientes (se pueden ver desde el anexo \ref{a:estruc_f} hasta el anexo \ref{a:estruc_d}):
\begin{itemize}
    \item Objeto amarillo dentro de la torre hidropónica representa la bomba de agua sumergible
    \item Los panales morados representan los leds de cultivos de plantas
    \item Cajas de color verde y rojo representan el arduino y protoboard respectivamente dentro del casing.
\end{itemize}

\subsubsection{Página Web}
Para la implementación de la página web se planea utilizar un conjunto de
tecnologías relacionadas con el <<cloud storage>>, entre estas se tiene la base
de datos para el almacenamiento de la información obtenida por los censores
físico y el servidor para poder hacer publica la página web; para estos
servicios se tiene como primera opción utilizar los servicios de
\href{https://aws.amazon.com/}{AWS}.

El diseño que se puede observar en el anexo \ref{a:wireframe}. Esté wireframe
se plantea de forma simple con pocos botones y un panel que dependiendo de la
opción seleccionada muestra el los datos actuales que los diferentes censores
recolectan. La simpleza es a razón de que no se desea que aquel que interactúe
con el prototipo tenga que dedicar mucho tiempo para entender como funciona la
interfaz.

\subsubsection{Página Web}
El diagrama de bloques se puede observar el en anexo \ref{a:diag_bloques}.
