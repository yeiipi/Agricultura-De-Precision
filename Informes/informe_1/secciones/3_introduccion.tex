%%%%%%%%%%%%%%%%%%%%%%%%%%%%%%%%%%%%%%%%%%%%%%%%%
\section{Introducción} % NOMBRE DE LA SECCIÓN
\label{sec:introduccion} % ETIQUETA
%%%%%%%%%%%%%%%%%%%%%%%%%%%%%%%%%%%%%%%%%%%%%%%%%
\thispagestyle{plain}

\noindent Actualmente los cultivos tradicionales se encuentran con múltiples
adversidades debido a la afectación por cambios climáticos, pues todos los
cultivos requieren de determinadas características para su cosecha como lo es
la temperatura, el suelo o localización geográfica pues depende de las
condiciones del suelo y su fertilidad para la debida prosperidad del cultivo;
durante años se ha implementado el cultivo tradicional por épocas del año y se
dificulta el cultivo en ciudades principales e islas con escasez de tierra
fértil, pero actualmente se evidencian problemas por las necesidades de las
personas en el consumo de estos y de las ganancias de las empresas, por esta
razón se comenzó a implementar los cultivos hidropónicos dado que se puede
cultivar en todo momento, bajo condiciones controladas en espacios reducidos y
en cualquier lugar, parte de ahí la tecnología comienza a desarrollar este
sector de agricultura controlando condiciones y cumpliendo con los
requerimientos como parte importante para poder aumentar la precisión en el
monitoreo y mantenimiento de los cultivos (\cite{katama2019}).

\noindent Se identifica como problema en el proyecto agrotech, la falta de
espacio, tierra fértil y tiempo de las personas para dedicarle a un huerto
casero, además de su dificultad en saber las condiciones y cuidados necesarios
de cada planta que se quiera tener en la huerta. Para esto se propone el
desarrollo de un huerto casero inteligente automatizado por medio de sensores
basado en un sistema de cultivo hidropónico enlazado a un aplicativo web para
facilitar la visualización de las variables de los cultivos en tiempo real.

\noindent Un cultivo hidropónico es una técnica en el que las plantas pueden
crecer sin necesidad de tener tierra o un suelo determinado (\cite{belen2019}).
Es un sistema basado en agua y disoluciones minerales donde se aplican todos
los nutrientes necesarios para el cultivo, este sistema es poco usual para
desarrollar huertos caseros, sin embargo esta maximiza la prosperidad de los
cultivos dado por sus propiedades y estructura, esta funciona por medio de
tuberías y canales en modo vertical donde la disolución de minerales está en
constante circulación, de
este modo el agua no se estanca y el cultivo se nutre a su propia necesidad
(\cite{twenergy2019}). La estructura de este cultivo hidropónico al ser en modo vertical reduce
el espacio ocupado en relación a los huertos normales, en contraste, un huerto
normal es horizontal y debe haber espacio suficiente entre plantas para que su
raíz no afecte el rendimiento de las demás por esta razón la estructura de un
huerto casero basado en un sistema hidropónico optimiza el espacio y
rendimiento del cultivo a realizar (\cite{nievesperez2017}).

\noindent El desarrollo del sistema en el huerto casero inteligente
automatizado se propone mediante sensores de humedad, luz, temperatura y de ph
para controlar y medir los factores descritos por cada sensor, integrando los
por medio de la placa arduino, de modo que el sistema inteligente y
automatizado se activa teniendo en cuenta las condiciones específicas de los
cultivos a manejar en la huerta y de acuerdo a las condiciones predispuestas y
programadas por cada sensor, dando la señal correspondiente para dar paso al
proceso que se requiere y además su respectiva visualización en el aplicativo
web.

\noindent La importancia de automatizar el huerto casero radica en que
automatizar procesos como el riego, exposición adecuada y necesaria de luz, la
estabilización de ph (por medio de la disolución de minerales), estabilizar la
temperatura y controlar la humedad se integran y sustituyen los procesos
manuales, dejando así como beneficios, acelerar el tiempo de realizar estos
procesos, generar una disminución de consumo de recursos brindando exactitud en
cantidades destinadas a cada proceso, disminuir los posibles errores tales como
la falta de tiempo, desconocimiento y descuido de los cultivos y tener
seguimiento oportuno para un mejor control y desarrollo de los procesos (\cite{ricopia2018}).
