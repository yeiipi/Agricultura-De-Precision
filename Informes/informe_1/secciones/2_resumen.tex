\small
\thispagestyle{plain}

\vspace{11pt}

\centerline{\rule{0.95\textwidth}{0.4pt}}

\begin{center}

    \begin{minipage}{0.9\textwidth}
        % RESUMEN
        \textbf{Resumen:} El proyecto agrotech es un proyecto basado en la
        inclusión de innovación y tecnología en la agricultura teniendo en
        cuenta las necesidades de tiempo para manejarlos, condiciones
        específicas, control y espacio, para la solución de estos factores se
        optó por un proyecto inicial de huerto casero automatizado por medio de
        sensores en un sistema de cultivo hidropónico. Se determinó realizarse
        en un sistema hidropónico dadas las características de espacio y buen
        manejo de las condiciones de los cultivos, pues según estudios con un
        sistema así los cultivos crecen de una manera más óptima, en cuanto a
        su estructura es de forma vertical por lo cual ocupa menos espacio y
        visiblemente es más armonioso para una vivienda, su funcionamiento es
        por medio de agua en canales, de esta manera cada planta absorbe el
        porcentaje de agua necesario y así mismo pueden fluir disoluciones de
        minerales, abonos, plaguicida y demás componentes que se quieran
        disponer para el buen desarrollo del cultivo (\cite{culthidrop2020}).

        \noindent En cuanto a los sensores a utilizar para el proyecto se
        incorporan las siguientes variables: luz, temperatura, ph y humedad,
        los cuales van a ser alimentados por medio de estudios e
        investigaciones previas realizadas de cada planta que pueda manejarse
        en un cultivo hidropónico estas plantas a estudiar son incluidas en una
        base de datos, para tener segmentados los cultivos, las características
        propias de cada una y sus debidas condiciones. Se van a determinar las
        condiciones óptimas de cada variable por planta para así poder enlazar
        los sensores con el cultivo y el software realizado y lograr activar
        oportunamente el requerimiento que se programe por cada condición,
        adicionalmente estas condiciones y requerimientos ya automatizados se
        pueden observar para mayor tranquilidad, control y seguridad por quien
        sea el responsable del huerto por un aplicativo web, el aplicativo es
        brindado como una herramienta para ayuda del usuario donde sienta que
        esta interactuando con este sistema. El proyecto finalmente tiene una
        alta proyección puesto que es una base para poder incursionar en
        grandes cultivos comerciales.

        \vspace{4mm}
        % PALABRAS CLAVE
        \noindent \textbf{Palabras clave:} Cultivos hidropónicos, humedad, temperatura, pH, automatización.

    \end{minipage}

\end{center}

\centerline{\rule{0.95\textwidth}{0.4pt}}

\vspace{15pt}
